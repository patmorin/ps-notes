\documentclass{patmorin}
\listfiles
\usepackage{pat}
\usepackage{paralist}
\usepackage{dsfont}  % for \mathds{A}
\usepackage[utf8x]{inputenc}
\usepackage{skull}
\usepackage{paralist}
\usepackage{graphicx}
\usepackage[noend]{algorithmic}

\usepackage[normalem]{ulem}
\usepackage{cancel}
\usepackage{enumitem}

\newcommand{\defin}[1]{\textcolor{Maroon}{\emph{#1}}}

\usepackage{todonotes}

\usepackage[longnamesfirst,numbers,sort&compress]{natbib}

\usepackage[mathlines]{lineno}
\setlength{\linenumbersep}{2em}
% \linenumbers
% \rightlinenumbers
% \linenumbers
\newcommand*\patchAmsMathEnvironmentForLineno[1]{%
 \expandafter\let\csname old#1\expandafter\endcsname\csname #1\endcsname
 \expandafter\let\csname oldend#1\expandafter\endcsname\csname end#1\endcsname
 \renewenvironment{#1}%
    {\linenomath\csname old#1\endcsname}%
    {\csname oldend#1\endcsname\endlinenomath}}%
\newcommand*\patchBothAmsMathEnvironmentsForLineno[1]{%
 \patchAmsMathEnvironmentForLineno{#1}%
 \patchAmsMathEnvironmentForLineno{#1*}}%
\AtBeginDocument{%
\patchBothAmsMathEnvironmentsForLineno{equation}%
\patchBothAmsMathEnvironmentsForLineno{align}%
\patchBothAmsMathEnvironmentsForLineno{flalign}%
\patchBothAmsMathEnvironmentsForLineno{alignat}%
\patchBothAmsMathEnvironmentsForLineno{gather}%
\patchBothAmsMathEnvironmentsForLineno{multline}%
}


\newcommand{\coloured}[2]{{\color{#1}{#2}}}
\newenvironment{colourblock}[1]{\color{#1}}{}

\newcommand{\condref}[1]{(C\ref{#1})}

% Taken from
% https://tex.stackexchange.com/questions/42726/align-but-show-one-equation-number-at-the-end
\newcommand\numberthis{\addtocounter{equation}{1}\tag{\theequation}}


\setlength{\parskip}{1ex}


\DeclareMathOperator{\diam}{diam}
\DeclareMathOperator{\tw}{tw}
\DeclareMathOperator{\gm}{gm}
\DeclareMathOperator{\gs}{gs}
\DeclareMathOperator{\stw}{stw}
\DeclareMathOperator{\ltw}{ltw}
\DeclareMathOperator{\pw}{pw}
\DeclareMathOperator{\lpw}{lpw}
\DeclareMathOperator{\lhptw}{lhp-tw}
\DeclareMathOperator{\lhppw}{lhp-pw}

\DeclareMathOperator{\x}{x}
\DeclareMathOperator{\depth}{d}
\DeclareMathOperator{\sh}{cbt}
\DeclareMathOperator{\cbt}{cbt}
\DeclareMathOperator{\sgn}{sgn}
\DeclareMathOperator{\dc}{dc}
\DeclareMathOperator{\afci}{\overline{\chi}_\pi}
\DeclareMathOperator{\afcn}{\dot{\chi}_\pi}

\newcommand{\ellt}{{\lfloor\ell/2\rfloor}}

\title{\MakeUppercase{Notes on Product Structure}\thanks{This research was partly funded by NSERC.}}
\author{Vida Dujmović%
    \thanks{Department of Computer Science and Electrical Engineering, University of Ottawa}\qquad
    Pat Morin%
    \thanks{School of Computer Science, Carleton University}}

\date{}

\DeclareMathOperator{\ddiv}{div}
\DeclareMathOperator{\hist}{h}

\newcommand{\colored}[2]{{\color{#1}#2}}

\usepackage{tabularx}

\DeclareMathOperator{\ci}{\overline{\pi}}

\begin{document}

\begin{titlepage}
\maketitle

\begin{abstract}
  This is a collection of notes and observations about product structure.
\end{abstract}
\end{titlepage}

\pagenumbering{roman}
\tableofcontents

\newpage
\pagenumbering{arabic}



\section{Introduction}

This is a collection of notes and observations about product structure.

\newpage

\section{Product Structure Does Not Imply Subquadratic Grid Minors}

Let $\boxplus_g$ denote the $g\times g$ grid graph.
For a graph $G$, let $\tw(G)$ denote the treewidth of $G$ and let $\gm(G)$ denote the largest value $g$ such that $G$ contains a $\boxplus_g$ minor.
A family of graphs $\mathcal{G}$ has the \defin{subquadratic grid minor property (SQGM) property} if there exists a constant $\alpha =\alpha(\mathcal{G}) > 1/2$ such that $\gm(G)\in\Omega(\tw(G)^{\alpha})$ for every $G\in\mathcal{G}$. For example, every planar graph $G$ contains an $\Omega(\tw(G))\times\Omega(\tw(G))$ grid minor, so $\gm(G)\in\Omega(\tw(G))$, so the family of planar graphs has the SQGM Property (with $\alpha=1$).
% (Equivalently, $\tw(G)\in O(\gm(G)^{1/\alpha})$.
The SQGM property is extremely useful from an algorithmic perspective and allows many so-called \emph{bidimensional problems} to be solved efficiently on any graph family with the SQGM property \cite{fomin.lokshtanov.ea:excluded}.

% A family of graphs $\mathcal{G}$ has the \defin{subquadratic grid minor property (SQGM)} if there exists constants $\alpha, c$, $\alpha >0$, $1\le c< 2$ such that, for any $t>0$ any graph $G\in\mathcal{G}$ that does not have a $t\times t$ grid minor has treewidth at most $\alpha t^c$.  Equivalently, any $G\in\mathcal{G}$ contains an $\Omega(\tw(G)^{1/c})\times \Omega(\tw(G)^{1/c})$ grid minor.

% A more recently discovered property of planar graphs is that, for any planar graph $G$ there exists a graph $H$ of treewidth at most $8$ and a path $P$ such that $G$ is a subgraph of $H\boxtimes P$. This \defin{product structure property} of planar graphs has been used to solve a number of longstanding open problems on planar graphs \todo{references}.  Moreover, these solutions only use this product structure property which means they also apply to a number of non-planar graph classes including bounded genus graphs, apex-minor free graphs, bounded-degree graphs from minor-closed families, and $k$-planar for fixed $k$.\todo{references}

Let $\mathcal{G}_t:=\{G:G\subseteq H\boxtimes P,\ \tw(H)\le t\}$, i.e., the subgraph-closed family of graphs that contains $H\boxtimes P$ for every graph $H$ of treewidth at most $t$ and every path $P$.  The original product structure theorem implies that $\mathcal{G}_8$ contains every planar graph.  It is natural to ask if, for every $t\in\N$, the class $\mathcal{G}_t$ has the SQGM property.  In this note we show that, unfortunately, this is not the case, even for the graph class $\mathcal{G}_1$.

The \defin{$n$-star} $S_n$ is the tree having $n+1$ nodes and $n$ leaves. The \defin{$n$-path} $P_n$ is the path on $n$ vertices.

\begin{thm}\label{lousy_theorem}
  $\tw(S_n\boxtimes P_n)\in \Omega(n)$ and $\gm(S_n\boxtimes P_n)\in O(\sqrt{n})$.
\end{thm}

Since $\{S_n\boxtimes P_n: n\in\N\}$ is included in $\mathcal{G}_1$ and $\mathcal{G}_1\subseteq \mathcal{G}_t$ for each $t\in\N$, \cref{lousy_theorem} implies that $\mathcal{G}_t$ does not have the SQGM property for any $t\in\N$.


\subsection{Proof of \cref{lousy_theorem}}

\cref{lousy_theorem} is an immediate consequence of \cref{star_gridminor} and \cref{star_treewidth}, proven below. First we need a some definitions and notations.  We let $x_0$ denote the root of $S_n$ and $x_1,\ldots,x_n$ denote the leaves of $S_n$.  We let $y_1,\ldots,y_n$ denote the vertices of $P_n$, in the order they occur on $P_n$.

For a graph $G$ that contains a graph $M$ as a minor, a \defin{model} of $M$ in $G$ is a sequence $(B_x:x\in V(M))$ of vertex-disjoint connected subgraph of $G$, indexed by the vertices of $M$ such that, for every edge $xy\in E(M)$ there exists an edge $vw\in E(G)$ with $v\in V(B_x)$ and $w\in V(B_y)$.  By definition, if $G$ contains $M$ as a minor, then there exists a model of $M$ in $G$.

\begin{lem}\label{star_gridminor}
  $\gm(S_n\boxtimes P_n)\le \sqrt{5n}$
\end{lem}

\begin{proof}
  Suppose that $S_n\boxtimes P_n$ contains a $\boxplus_k$ grid minor.  We will prove that $k\le\sqrt{5n}$.  Let $\mathcal{M}:=(B_x:x\in V(\boxplus_k))$ be a model of $\boxplus_k$ in $G$.  We say that a piece $B_x$ of $\mathcal{M}$ is \defin{rooty} if $(x_0,y)\in V(B_x)$ for some $y\in V(P)$ and $B_x$ is \defin{rootless} otherwise.  Clearly, the number of rooty parts in $\mathcal{M}$ is at most $n$.

  Observe that each rootless part $B_x$ of $\mathcal{M}$ is a single path $(x_i,y_j),\ldots,(x_i,y_{j+\ell})$.  There are at most two other rootless parts [one containing $(x_i,y_{j-1})$ and one containing $(x_i,y_{j+\ell+1})$] adjacent to $B_x$.  However, the vertex $x$ has degree at least $3$ in $\boxplus_k$, so $B_x$ must be adjacent to at least one rooty part $B_y$ such that $xy$ is an edge of $\boxplus_k$; we call this a \defin{rooty-rootless adjacency}. The model $\mathcal{M}$ has $k^2$ parts and at most $n$ of these are rooty, so there are at least $k^2-n$ rootless parts.  Therefore, there are at least $k^2-n$ rooty-rootless adjacencies.  Each rooty part of $\mathcal{M}$ can account for at most $4$ of these adjacencies, so $k^2-n \le 4n$, proving the result.\footnote{We could actually prove a bound of $\approx\sqrt{2n}$ by using the fact that most rootless parts need to be adjacent to $2$ rooty parts.}
\end{proof}

\begin{lem}\label{star_treewidth}
  $\tw(S_n\boxtimes P_n) \ge (n-4)/3$
\end{lem}

\begin{proof}
  Let $G:=S_n\boxtimes P_n$ and suppose that $\tw(G)=t$.  We will show that $t\ge (n-4)/3$. Since $\tw(G)=t$, there exists a \defin{separator} $X\subset V(G)$ of size at most $t+1$ such that $G-X$ has no component of size greater than $2|V(G)|/3=2n(n+1)/3$.  We partition the separator $X$ into two parts $X_0:=X\cap(\{x_0\}\times V(P_n)$ and $Y:=X\setminus X_0=X\cap(\{x_1,\ldots,x_n\}\times V(P_n))$.

  For each $i\in\{0,\ldots,n\}$, let $P_i$ denote the path $(x_i,y_1),\ldots,(x_i,y_n)$.  Let $I:=\{i\in\{1,\ldots,n\}:Y\cap\{x_i\}\times V(P_n)\neq\emptyset\}$, let $\overline{I}:=\{1,\ldots,n\}\setminus I$, let $Y^+:= I\times\{y_1,\ldots,y_n\}$ and consider the subgraph $G':=G-Y^+$.

  The separator $X$ must also be a separator of $G'$ in the sense that $G'-X$ has no component of size greater than $2|V(G)|/3$.  However, the vertices of $Y$ are not in $G'$, so $G'-X=G'-X_0$.  For any $i\in\overline{I}$, the entire path $P_i$ is contained in $G'-X_0$.

  If $\{x_0\}\times \{y_1,\ldots,y_n\}\setminus X_0=\emptyset$, then $n=|X_0|\le t+1$, so $t\ge n-1\ge (n-4)/3$ and we are done.  Otherwise there exists some $(x_0,y_j)\in \{y_1,\ldots,y_n\}\setminus X_0$.  The vertex $(x_0,y_j)$ is in $G'$ and is adjacent, in $G'$, to the vertex $(x_i,y_j)$ of $P_i$, for each $i\in\overline{I}$.  Therefore, there is a single component of $G'-X_0$ that contains $\bigcup_{i\in \overline{I}} V(P_i)$.  This component has size at least
  \[
    n\cdot|\overline{I}| \ge n(n-t-1)
  \]
  But this component must have size at most $2n(n+1)/3$, so
  \[
    n(n-t-1) \le 2n(n+1)/3 \enspace ,
  \]
  % (n-t-1) \le 2(n+1)/3  [divide by n]
  % (n-2n/3 - 4/3) \le t  [add t - 2n/3 - 1/3]
  % (n/3 - 4/3) \le t     [simplify]
  % (n-4)/3 \le t         []
  and rewriting this implies that $t\ge (n-4)/3$.
\end{proof}

\begin{rem}
  Note that \cref{lousy_theorem} does not even use the fact that $\mathcal{G}_t$ is closed under taking subgraphs.
\end{rem}

\begin{rem}
  Note that the proofs of \cref{star_gridminor} and \cref{star_treewidth} do not use diagonal edges from the strong product, so they hold also for the Cartesian product.
\end{rem}


\subsection{Tightness}

Can we prove that \cref{lousy_theorem} is tight?  Is it true that every $G\in \mathcal{G}_t$ contains a $\boxplus_k$ minor for $k\in\Omega(\sqrt{\tw(G)})$?  In general, the strongest \emph{Excluded Grid Theorem} for geneneral graphs states that any graph $G$ contains a $\boxplus_{k}$ minor for $k\in\tilde{\Omega}(\tw(G)^{1/9})$ \cite{chuzhoy.tan:towards}.  So we're asking for a much stronger Excluded Grid Theorem for graphs in $\mathcal{G}_t$.

Here are two results proving the existence of large grid minors (actually subdivisions) in graph products.  Some of what comes next might be helpful, but remember that $\mathcal{G}_t$ doesn't just contain products, it also contains all their subgraphs.  For a graph $G$, let $\gs(G)$ be the largest value $k$ such that $G$ contains a subdivision of $\boxplus_k$.  Clearly $\gm(G)\ge\gs(G)$.

% Does there exists a constant $C$ such that $\gm(H\boxtimes P)=k$ implies that $

\begin{lem}\label{star_times_path}
    Let $P:=P_{n^2-n+1}$ be a path of length $n^2-n+1$, let $H$ be a graph, and let $x_1,\ldots,x_n$ be a sequence of distinct vertices in $H$ such that, for each $i\in\{1,\ldots,n-1\}$, $H$ contains a path $P_{i}$ from $x_i$ to $x_{i+1}$ that does not contain any vertex of $\{x_1,\ldots,x_n\}$ in its interior.  Then $\gs(H\boxtimes P)\ge n$.
\end{lem}

\begin{proof}
    Let $G=H\boxtimes P$ and let $P:=y_1,\ldots,y_{n^2-n+1}$.  Our goal is to find a subdivision of $\boxplus_n$ inside of $G$.  To do this, The vertex $(i,j)\in\boxplus_n$ will map to the vertex $v_{i,j}:=(x_i,y_{i+(j-1)n})\in V(G)$.  For each $i\in\{1,\ldots,n\}$, the path $(i,1),(i,2),\ldots,(i,n)$ in $\boxplus_n$ will map to the subpath of $(x_i,y_1),\ldots,(x_i,y_{n^2-n+1})$ that begins at $v_{i,1}$ and ends at $v_{i,n}$.

    Now consider some edge $(i,j)(i+1,j)$ of $\boxplus_n$ and let $P_i:=w_{i,0},\ldots,w_{i,d_i}$ be a path in $G$ from $w_{i,0}:=x_i$ to $w_{i,d_i}:=x_{i+1}$ that avoids $\{x_1,\ldots,x_n\}$.  Then the horizontal edge $(i,j)(i+1,j)$ maps onto the path
    \[
        (v_{i,0},y_{i+(j-1)n})(v_{i,1},y_{i+1+(j-1)n}),\ldots,(v_{i,d_i},y_{i+1+(j-1)n}) \enspace .
    \]
    where $v_0,\ldots,v_d$ is any path in $G$ from $x_{i}$ to $x_{i+1}$.
\end{proof}

\cref{star_times_path} did surprise me  because the condition on $H$ is satisfied even when $H$ is a star with leaves $x_1,\ldots,x_n$.  In fact, it's satisfied when $H$ is any tree with $n$ leaves.  Actually, if we take any connected graph $H$ with $n$ vertices and take any spanning tree $T$ of $H$, we get a tree that has at least $\sqrt{n}$ leaves or that has a path of length at least $\sqrt{n}$.  This gives the following corollary of \cref{star_times_path}:

\begin{cor}\label{graph_times_path}
  Let $H$ be any connected graph with $n$ vertices and let $P$ be a path of length $n$.  Then $\gs(H\boxtimes P)\ge \sqrt{n}$.
\end{cor}

% So \cref{star_times_path} implies that $\gm(H\boxtimes P)\ge\gs(H\boxtimes P)\ge\sqrt{\tw(H\boxtimes P)}$.  To get the subquadratic grid minor property we would only have to squeeze  an extra $\epsilon$ out of this argument.  Where could we squeeze out this extra $\epsilon$?  There is an asymmetry between the two cases in \cref{graph_times_path}:  If $H$ contains a path of length $\ell$, then $H\boxtimes P_\ell$ contains $\boxplus_\ell$ as a subgraph, so $\gs(H\boxtimes P_\ell)\ge \ell$.  On the other hand, if $H$ is a star with $\ell$ leaves, then \cref{graph_times_path} only implies that $\gs(H\boxtimes P_\ell)\ge \sqrt{\ell}$.  Is \cref{graph_times_path} tight in the second case?  Yes:
%
% \begin{lem}\label{star_times_path_tightness}
%     For any star $S$ and any path $P$ with $\ell$ vertices, $\gm(S\boxtimes P)\le 2\sqrt{\ell}$.
% \end{lem}
%
% \begin{proof}
%     Let $G:=S\boxtimes P$ and suppose that $G$ contains a $\boxplus_k$ grid minor.  Let $\mathcal{M}:=(X_x:x\in V(\boxplus_k))$ be the model of $\boxplus_k$ in $G$.  Let $r$ denote the root of $S$.  We say that a piece $X_x$ is \defin{rooty} if $(r,y)\in V(X_x)$ for some $y\in V(P)$ and \defin{rootless} otherwise.  Clearly, the number of rooty parts in $\mathcal{M}$ is at most $\ell$.
%
%     Each rootless part in $\mathcal{M}$ is adjacent to at most two other rootless parts.  However, each part of $\mathcal{M}$ is adjacent to at least three other parts, so each rootless part is adjacent to at least one rooty part.  Each rooty part can account for at most $4$ of these adjacencies, so $k^2 \le 4\ell$, proving the result.\footnote{We could actually prove a bound of $\sqrt{2\ell}$ by using the fact that most rootless parts need to be adjacent to $2$ rooty parts.}
% \end{proof}
%
% Again, \cref{star_times_path_tightness} doesn't prove or disprove anything, unless we can show that $\tw(S\boxtimes P)\in \Omega(\ell)$.  I suspect that this is indeed the case, so let's try to prove it.
%
% \begin{lem}
%     For any star $S$ with $n$ leaves and any path $P$ with $n$ vertices, $\gm(S\boxtimes P)\in\Omega(\ell)$.
% \end{lem}
%
% \begin{proof}
%     Let $G:=S\boxtimes P$ and suppose that $\tw(G)=k\in o(n)$.  Let $P:=y_1,\ldots,y_n$, let $x_0$ be the root of $S$ and let $x_1,\ldots,x_n$ be the leaves of $S$. Since $\tw(G)=k$, there exists a separator $X\subset V(G)$ of size $k+1$ such that $G-S$ has no component of size greater than $2|V(G)|/3=2n(n+1)/3$.  Let $X_S:=\{x:(x,y)\in X,\, x\in\{1,\ldots,n\}\}$.  Consider the subgraph $G':=G-X_S$, which has $|V(G')|\ge n(n+1)-nk \in n(n+1)-o(n^2)$ vertices.  The separator $X$ must also be a separator of $G'$ in the sense that $G-X$ has no component of size greater than $2|V(G)|/3$.
%
%
% \end{proof}



% Alternatively, if $G$ has treewidth at least $\alpha t^c$ ($k$) then $G$ contains a $t\times t$ grid minor.

We can also ask if the SQGM property implies product structure:
 
\begin{op}
  Let $\mathcal{G}$ be a graph family with the property that, for some fixed $\epsilon >0$, $\gm(G)\in\Omega(\tw(G)^{1/2+\epsilon})$, for each $G\in\mathcal{G}$.  Does this imply that that, for each $G\in\mathcal{G}$, $G\subseteq H\boxtimes P$ for some graph $H$ of treewidth at most $t:=t(\mathcal{G})$ and some path $P$?
\end{op}

One could start by considering the (easier?) case in which each $\epsilon = 1/2$.


\newpage
\section{Disk Graphs of Bounded Local Radius Have Product Structure}

\citet{lokshtanov.panolan:framework} introduce the concept of the local radius of disk graphs.  They show that a number of optimization problems on arbitrary disk graphs can be efficiently reduced (with an $\epsilon$-approximation error) to the same problem on a disk graph of bounded local radius.

Let $\mathcal{D}$ be a set of disks in the plane and let $A$ be the \defin{(dual) arrangement} graph of these disks.  That is, $V(A)$ consists of all the connected components of $\R^2\setminus(\bigcup_{D\in\mathcal{D}}\partial D)$ and $fg\in E(A)$ if there is exactly one disk $D\in\mathcal{D}$ such that $f\subseteq D$ and $g\cap D=\emptyset$.  For any $D\in\mathcal{D}$ let $A[D]:=A[\{f\in V(A): f\subseteq D\}]$ be the subgraph of $A$ induced by the regions contained in $D$.  The set $\mathcal{D}$ has \defin{local radius} at most $r$ if the radius of $A[D]$ is at most $r$ for each $D\in\mathcal{D}$.

The \defin{intersection graph} $G$ of $\mathcal{D}$ is a graph with $V(G)=\mathcal{D}$. To provide a conceptual distinction between vertices of $G$ and disks in $D$ we will use, for example $v$ to refer to a vertex of $G$ and $D_v$ to refer to the disk in $\mathcal{D}$.  An edge $vw$ is in the intersection graph $G$ if and only if $D_v\cap D_w\neq\emptyset$.  If the et $\mathcal{D}$ of disks that defines $G$ has local radius at most $r$ then we say that $G$ has \defin{local radius} at most $r$.  We would like to prove the following result:

\begin{thm}\label{radius_product}
   If $G$ is a disk intersection graph of local radius at most $r$ then  $G\subseteq H\boxtimes P\boxtimes K_{O(r^2)}$ where $\tw(H)\in O(r^3)$ and $P$ is a path.
\end{thm}

Our starting point is the following result, which is a specialization of \citet[Theorem~7]{hickingbotham.wood:shallow}:

\begin{thm}\label{h_w_shallow_minor}
  If $G$ is a subgraph of $H\boxtimes P\boxtimes K_c$ and $G'$ is an $r$-shallow minor of $G$ then $G'$ is a subgraph of $H' \boxtimes P \boxtimes K_{O(cr)}$ where $\tw(H') \in O(r^{\tw(H)})$.
\end{thm}

The \defin{ply} of a set $\mathcal{D}$ of disks is $\max_{p\in\R^2}|\{D\in\mathcal{D}:p\in D\}|$.  We first show that bounded local radius implies bounded ply.

\begin{lem}\label{ply}
  If a set $\mathcal{D}$ of disks has local radius at most $r$ then the ply of $\mathcal{D}$ is at most $2r+1$.
\end{lem}

\begin{proof}
  Suppose some set $S\subseteq\mathcal{D}$ of $k$ disks have a point $p\in \R^2$ in common.  The boundary of the union of disks in $S$ is a collection of circular arcs.  Let $q$ be a point in the interior of one of these arcs.  Then $q$ is on the boundary of exactly one disk $D$ in $S$.  Any curve with endpoints $p$ and $q$ must cross the boundaries of the $k-1$ disks in $S \setminus \{D\}$, so the diameter of $A[D]$ is at least $k-1$.  Therefore the radius of $A[D]$ is at least $(k-1)/2$ so $(k - 1)/2 \le r$ , so $k \le  2r+1$.
\end{proof}

\begin{proof}[Proof of \cref{radius_product}]
  Let $\mathcal{D}$ be a set of disks of local radius $r$, let $A$ be the arrangement graph of $\mathcal{D}$ and let $G$ be the intersection graph of $\mathcal{D}$. Without loss of generality we may assume that no two disks in $\mathcal{D}$ have the same center, since a slight perturbation of the disks will not change the graph $G$.

  For $v\in V(G)$ let $Y_v:=A[D_v]$.  Since $\mathcal{D}$ has local radius $r$,  the radius of $Y_v$ is at most $r$, for each $v\in V(G)$. By \cref{ply}, for each  $f\in V(A)$, there are at most $2r+1$ vertices $v\in V(G)$ such that $f\in V(Y_v)$.  For each $f\in V(A)$, choose any injective function $\phi_f: \{v \in V(G): f \in V(Y_v)\}\to\{1,...,2r+1\}$.

  Now consider the graph $Q:=A \boxtimes K_{2r+1}$ where the vertices of $K_{2r+1}$ are the integers $\{1,...,2r+1\}$.  For each $v\in V(G)$, define $Y'_v$ as the subgraph of $Q$ induced by $\{(f, \phi_f(v)): f \in V(Y_v)\}$.  Since $Y_v$ is connected, so is $Y'_v$. For two distinct $v,w \in V(G)$, $Y_v$ and $Y_w$ are not necessarily disjoint.  However, the fact that $\phi_f$ is injective, for each $f\in V(A)$ implies that $Y'_v$ and $Y'_w$ are disjoint.  Furthermore, if $vw$ is an edge of $G$ then $Q$ contains an edge between $Y'_v$ and $Y'_w$.  Indeed, since $vw$ is an edge of $G$, $D_v$ and $D_w$ intersect, therefore $V(Y_v)\cap V(Y_w)\neq\emptyset$. Let $f$ be any element of $V(Y_v)\cap V(Y_w)$.  By the definition of strong product,  $(f,\phi_f(v))(f,\phi_f(w))$ is an edge of $Q$.

  Therefore $\{Y'_v : v \in V(G)\}$ is an $r$-shallow model of (a supergraph of) $G$ in $Q$.   $A$ is a planar graph, so $A\subseteq H\boxtimes P\boxtimes K_3$ where $\tw(H)\le 3$. Therefore $Q=A\boxtimes K_{2r+1}\subseteq H\boxtimes P\boxtimes K_{6r+3}$ for same graph $H$ of treewidth at most $3$.  By \cref{h_w_shallow_minor}, $G$ is a subgraph of $H' \boxtimes P \boxtimes K_{O(r^2)}$ where $\tw(H') = O(r^3)$.
\end{proof}


\bibliographystyle{plainurlnat}
\bibliography{ps-notes}


\end{document}
